\documentclass{article}

\usepackage{amsmath}
\usepackage{amssymb}

\author{Aurora Zuoris \\ \texttt{aurora.zuoris101@alu.ulpgc.es}}

\title{Sparce matrix multiplication}

\begin{document}

\maketitle

\abstract{
Sparce matrices are matrices that have a lot of zero elements,
thus it is possible to save a lot of memory by not storing these zero elements.
This can be beneficial in many applications, such as graph theory, where the adjacency matrix of a graph is sparce.
This paper will explore the different ways of storing sparce matrices and the different ways of multiplying them,
mainly focusing on the CCS and coordinate formats.
}

\section{Introduction}

\section{Implementations}

\section{Results}

\section{Conclusion}

\section{Future work}

\end{document}